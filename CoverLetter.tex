Dear Editor:

Enclosed please find the manuscript ``Title'' by Eric S. Shamay and Geraldine L. Richmond, submitted for consideration for publication in \textit{The Journal of Physical Chemistry B}.

The scientific community's understanding of the behavior of simple, inorganic ions at an organic-water interface is currently sparse as such systems have remained largely unexplored. Great advances have been made in studying ions at air-water interfaces over the last few years, however much can still be learned in the study of other surfaces. Computational analysis of molecular dynamics simulations is well suited for the study of narrow interfacial regions due to its ability to address microscopic environments and to probe the origins of specific phenomena that are impossible through current experimental techinques.

In this study we have shown that ions affect the structure and hydrogen bonding of water at the liquid-liquid interface in a very different manner than has been previously seen at the air-water interface. We augment a recent experimental sum frequency spectroscopic study by our group and strengthen the conclusions of that work while also building a more complete microscopic picture of the interfacial region. The alteration of water in comparison to the interfacial environment without ions is quite drastic. We find that different ions affect the interfacial aqueous layer differently. Our work will be of interest to the readership of \textit{The Journal of Physical Chemistry B} because it contributes significantly to our understanding of how the subtle interplay between water and an adjacent hydrophobic liquid uniquely promotes the solvation and accumulation of ions within the liquid-liquid interface, a subject that has implications in many important processes such as ion transport and interfacial catalysis.

I am the corresponding author and may be contacted at the above address, by e-mail (richmond@uoregon.edu), telephone (541-346-4635), or fax (541-346-5859).  All authors have seen and approved the submission of this manuscript, and we look forward to the results of the review.

 

Thank you for your consideration.

 

Sincerely,  
Geraldine Richmond

Richard M. and Patricia H. Noyes Professor
