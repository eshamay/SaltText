\section{Introduction}

The most important biological and environmental processes depend on the nature of interfacial water molecules and ions. Only in recent years, and through the development of surface-specific experimental\cite{Charreteur2008,Chen2007,Luo2006,McArthur2006} and computational\cite{Schnell2004,Su2005,Wardle2005,Wick2008a} analytical techniques, have we been able to begin understanding this complex environment of interfacial water and ions. The field has moved from simple water systems in vacuum to studying ever more complex ones such as those near hydrophobic surfaces that are responsible for ion transport, liquid-liquid extraction, drug delivery, and environmental remediation. 

The computational studies presented herein have been conducted to augment recent experimental studies that showed how ions affect the interfacial region between an aqueous ionic solution and a hydrophobic liquid.\cite{McFearin2009} Classical molecular dynamics simulations were performed to analyze the interface formed between various aqueous salt solutions and \ctc. Three salt solutions were simulated, as well as a reference system consisting of neat-\wat for comparison to previous computational efforts.\cite{Hore2007,Hore2008,Hore2007a,Walker2006b,Walker2007a,Walker2007b} The salts used in the simulations were NaCl, NaNO$_3$, and Na$_2$SO$_4$. These were chosen for comparison to the recent experimental sum frequency generation (SFG) spectroscopy results for the same systems,\cite{McFearin2009} and to supplement those experiments with additional molecular-level information. Analyses were performed on the simulation data to extract ionic and molecular density data across organic interfaces, information about water orientation near the interfaces, and simulated SFG spectra. 

%The analyses are similar to, and logical extensions of previous computational work done on aqueous salt systems.\cite{Hore2008,Hore2007a,Hore2007,Wick2006c,Wick2007a,Wick2008,Walker2008}

The computational simulation results provide a microscopic molecular picture of the geometries and interactions occurring within these interfaces that is otherwise inaccessible with experimental methods. The results are in excellent agreement with those of the experimental study, while going further to provide a detailed picture of how water near the hydrophobic liquid surface is altered by the presence of different electrolyte ions, in the aqueous solution, that migrate into the interfacial region. Insights gained from these studies have important implications for understanding a host of environmental, biological, and technological processes involving water near hydrophobic surfaces.

%The results strongly support the experimental study, and provide the first molecular-level look at these water systems at a \ctc interface. Our work builds on years of studies aimed at understanding water's molecular behavior at various interfaces. This work confirms theoretically many of the conclusions we made from experiment, and also provides further microscopic information about aqueous salt systems.

