\section{Introduction}

The most important biological and environmental processes depend on the nature of interfacial water molecules and ions. Only in recent years, and through the development of surface-specific experimental\cite{Charreteur2008,Chen2007,Luo2006,McArthur2006} and computational\cite{Schnell2004,Su2005,Wardle2005,Wick2008a} analytical techniques, have we been able to begin understanding this complex environment of interfacial water and ions. The field has moved from simple water systems in vacuum to studying ever more complex ones such as those near hydrophobic surfaces responsible for ion transport, liquid-liquid extraction, drug delivery, and environmental remediation.

The computational studies presented herein complement a recent experimental study by our group that showed how ions affect the interfacial region between an aqueous ionic solution and hydrophobic liquid carbon tetrachloride (CCl$_4$).\cite{McFearin2009} The computational simulation results provide a microscopic molecular picture of the geometries and interactions occurring within such interfaces that is otherwise inaccessible with experimental methods. The results strongly support the experimental study, and provide the first molecular-level look at these water systems at a \ctc interface. Our work builds on years of studies aimed at understanding water's molecular behavior at various interfaces. This work confirms theoretically many of the conclusions we made from experiment, and also provides further microscopic information about aqueous salt systems.

