\section{Introduction}

The most important biological and environmental processes depend on the nature of interfacial water molecules and dissolved ions in boundary layers. Only in recent years, and through the development of surface-specific experimental\cite{Charreteur2008,Chen2007,Luo2006} and computational\cite{Schnell2004,Wardle2005,Wick2008a} analytical techniques, have we been able to begin understanding this complex environment comprised of interfacial water and ions. Over this time the field has advanced from studying simple water systems in vacuum and in air, to studying more complex interfaces such as aqueous solutions near a hydrophobic surface that are responsible for such important processes as ion transport, liquid-liquid extraction, drug delivery, and environmental remediation. 

The computational studies presented herein have been conducted to compare and augment recent experimental studies that showed how ions affect the interfacial region between an aqueous ionic solution and a hydrophobic organic liquid.\cite{McFearin2009} Classical molecular dynamics simulations have been performed to analyze the interface formed between various aqueous salt solutions and the organic liquid carbontetrachloride (\ctc). Analyses were conducted to contrast the behavior of aqueous salt solutions as well as for comparison with previous computational efforts.\cite{Hore2007,Hore2008,Hore2007a,Walker2006b,Walker2007a,Walker2007b}  Three salt solutions were simulated containing \nacl, \sodnit, \sodsul. These were chosen to allow a comparison to the recent experimental sum frequency generation (SFG) results of these salt solutions,\cite{McFearin2009} and to supplement those experiments with additional molecular-level information derived from these simulations. The simulation data has been used here to extract ionic and molecular density data across organic interfaces, information about water orientation near the interfaces, and simulated SFG spectra. 

%as well as a reference system consisting of neat-\wat for comparison to previous computational efforts.

%The analyses are similar to, and logical extensions of previous computational work done on aqueous salt systems.\cite{Hore2008,Hore2007a,Hore2007,Wick2006c,Wick2007a,Wick2008,Walker2008}

The computational simulation results provide a microscopic molecular picture of the geometries and interactions occurring within these interfaces that is otherwise inaccessible with experimental methods. The general conclusions drawn from the simulations are in excellent agreement with those of the experimental study, while going further to provide a detailed picture of how the density and orientation of interfacial water near the hydrophobic liquid surface is altered by the presence of different electrolyte ions, in the aqueous solution, that migrate into the interfacial region. Insights gained from these studies have important implications for understanding a host of environmental, biological, and technological processes involving salt-containing water solutions near hydrophobic surfaces.

%The results strongly support the experimental study, and provide the first molecular-level look at these water systems at a \ctc interface. Our work builds on years of studies aimed at understanding water's molecular behavior at various interfaces. This work confirms theoretically many of the conclusions we made from experiment, and also provides further microscopic information about aqueous salt systems.

