\section{Abstract}

%The interface formed between aqueous salt solutions and hydrophobic liquid interfaces is of greatest importance in biological and environmental processes. The use of molecular dynamics simulations coupled with experimental analysis of these interfacial regions can provide a more complete understanding of the processes taking place within.

The region formed between aqueous salt solutions and a liquid hydrophobic phase was investigated using molecular dynamics simulations. \nacl, \sodnit, and \sodsul solutions were studied to determine their effect on the interfacial waters, and the resulting salt geometry near the surface region. Density and orientation profiles revealed formation of ionic double-layers of widths varying with respect to the anion's surface affinity, and affect on water's geometry. The \nit anion has an enhanced surface concentration above that of bulk, whereas the \cl and \sul anions exhibit similar characteristics as in corresponding air-water interfaces. Sum frequency spectra were calculated for the OH-vibrational modes of water to show the affect of the various ions on water's hydrogen-bonding network strength. These spectra also provide the necessary link to experiment, augmenting a previous experimental study, and are in very good agreement with the conclusions reached therein. It was found that \sodsul, forming the smallest ionic double-layer, acts to enhance the sum frequency spectra. The \sodnit system forms the widest double-layer and acts to diminish the sum frequency signal.
