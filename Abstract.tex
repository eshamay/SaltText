\section{Abstract}

%The region formed between aqueous salt solutions and a liquid hydrophobic phase was investigated using molecular dynamics simulations. \nacl, \sodnit, and \sodsul solutions were studied to determine their effect on the interfacial waters, and the resulting salt geometry near the surface region. Density and orientation profiles revealed formation of ionic double-layers of widths varying with respect to the anion's surface affinity, and affect on water's geometry. The \nit anion has an enhanced surface concentration above that of bulk, whereas the \cl and \sul anions exhibit similar characteristics as in corresponding air-water interfaces. Sum frequency spectra were calculated for the OH-vibrational modes of water to show the affect of the various ions on water's hydrogen-bonding network strength. These spectra also provide the necessary link to experiment, augmenting a previous experimental study, and are in very good agreement with the conclusions reached therein. It was found that \sodsul, forming the smallest ionic double-layer, acts to enhance the sum frequency spectra. The \sodnit system forms the widest double-layer and acts to diminish the sum frequency signal.

The interface formed between an aqueous salt solution and a hydrophobic liquid has been investigated using molecular dynamics simulations. The salt solutions, \nacl, \sodnit, and \sodsul have been studied to determine their presence and distribution in the interfacial region, and their effect on interfacial water molecules. Density and orientation profiles reveal the formation of ionic double-layers with widths that vary with the respective anions' surface affinities, and effects on the geometry of interfacial water molecules. The \nit anion shows enhanced surface concentration above that of the bulk aqueous phase, whereas the \cl and \sul anions exhibit similar characteristics as are found for corresponding air-water interfaces. Sum frequency spectra were calculated for the OH-vibrational modes of water to show the effect of the various ions on the hydrogen-bonding network strength of interfacial water. These calculated spectra show good agreement with the conclusions and observations of our recent spectroscopic experimental study, while providing important new detailed insights into interfacial behavior to augment that study.
