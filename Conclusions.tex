\section{Conclusions}

The unique environment created by interactions between water and hydrophobic molecules makes ionic adsorption and transport across interfaces possible. Complex interfaces between aqueous media and organic phases enhance chemical reactions, and this motivate further research to understand such environments. This study provides an important step in understanding aqueous-organic surfaces by computationally probing simple aqueous salt solutions interfaced with hydrophobic liquid \ctc. Through a combination of simulations and computational analysis, the nature of ionic adsorption and its effect on water hydrogen-bonding, geometry, and orientation at the liquid-liquid boundary is examined.

Analysis of the component density profiles provides a thorough microscopic picture of ionic surface affinity and effect on interfacial size. These analyses show that anion surface affinity is altered in the presence of an organic phase, and consequently affects the ability of water to penetrate into the hydrophobic region. Additionally, anion properties such as size, charge, and polarizability play roles in how the surface water interactions are enhanced. Further orientational analysis shows a stratifications of water geometries consistent with the emerging picture of a multi-layered surface region with varied geometries, interactions, and spectroscopic responses.

SFG spectra computed in this study build the necessary bridge to our previous SFG work by offering direct comparison of the computational and experimental results. We have thus moved toward our goal of further strengthening our previous experimental conclusions and developing this important and growing field of study. Our results here serve to motivate additional work on aqueous-hydrophobic interfaces. Such systems are found to be important in many applications ranging from ion transport, chemical remediation, and catalysis, to chemical synthesis at aqueous-organic interfaces. It is now clearer than ever that our quest for understanding these complex chemical environments necessitates further such studies.

%Simulations were conducted of the interface between a neat-\wat system, and three aqueous salt systems, and \ctc. By analyzing the density profiles of the water and the various ions in solution we found that both the water and ionic behavior at the liquid-liquid boundary is different than at the air-liquid one. Anions that have been found to be depleted at the air-water interface such as \nit have a strong interfacial affinity at the \ctcwat interface. Similarly, the orientation of water molecules is different at the liquid-liquid interface, and further altered when ions are present in solution. Depending on the ion surface affinity the waters orient to different extents and to different depths, supporting the conclusions of previous simulations and experiment. The field of interface simulation is further exposing the microscopic foundations of the macroscopic experimental results from techniques such as interface-specific spectroscopies. In our work we have connected the simulation results and methods by computing the SFG spectra to further complement previous experimental efforts. Further effort is still needed to provide a fuller understanding of the boundaries of aqueous systems and the effects of salt ions.
