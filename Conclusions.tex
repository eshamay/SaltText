\section{Conclusions}

Simulations were conducted of the interface between a neat-\wat system, and three aqueous salt systems, and \ctc. By analyzing the density profiles of the water and the various ions in solution we found that both the water and ionic behavior at the liquid-liquid boundary is different than at the air-liquid one. Anions that have been found to be depleted at the air-water interface such as \nit have a strong interfacial affinity at the \ctcwat interface. Similarly, the orientation of water molecules is different at the liquid-liquid interface, and further altered when ions are present in solution. Depending on the ion surface affinity the waters orient to different extents and to different depths, supporting the conclusions of previous simulations and experiment. The field of interface simulation is further exposing the microscopic foundations of the macroscopic experimental results from techniques such as interface-specific spectroscopies. In our work we have connected the simulation results and methods by computing the SFG spectra to further complement previous experimental efforts. Further effort is still needed to provide a fuller understanding of the boundaries of aqueous systems and the effects of salt ions.
