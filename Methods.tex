\subsection{Density Profiles}

Density histograms of simulated interfaces have been used in previous publications to show ionic and molecular distribution behavior in various systems.\cite{Chang1995,Eggimann2008,Du2008,Wick2006c,Petersen2005a,Hore2008,Walker2006b,Walker2007b} In this work the density profile of water throughout the interface is fitted to a hyperbolic tangent function\cite{Wick2006c,MATSUMOTO1988} as shown here:

\begin{equation}\label{tanh_fit}
	\rho(z) = \frac12(\rho_1+\rho_2) - \frac12\left(\rho_1-\rho_2\right)\tanh\left(\frac{z-z_0}{d}\right)
\end{equation}

Equation (\ref{tanh_fit}) relates the interfacial density, $\rho$, as a function of position, $z$, along a given system reference axis, to the densities of the phases on either side of the location of the Gibb's dividing surface (GDS), $z_0$. The bulk density $\rho_1$ and the density within the second phase, $\rho_2$ are fitted.  The interfacial width, $d$, is related to the ``90-10'' thickness that is often reported by $t_{90-10} = 2.197d$.

These measures of interfacial thickness provide a means of comparing the depths to which the water phase is affected by ions located at the interface. The density distributions of the salts depict concentration and depletion phenomena throughout the interfacial region, and also serve to illustrate ionic affinity within this region. Previous work has been performed on the \airwat interface with ions of different levels of interfacial affinity, with the more polar ions being the most interfacially active.\cite{Luo2006,Petersen2006,Petersen2005a,Allen2009,Hofft2006,Beattie2005,Bian2009,Dang2004b} We present the density distribution results below for the neat-H$_2$O and salt solutions adjacent to an organic CCl$_4$ phase. %The density profiles of the different ionic species are fitted using a modified $\tanh()$ function that includes a Gaussian function to more closely fit the concentration near the interface. The anion fitting function allows for a more direct comparison of location and peak width between the different systems.

%\begin{equation}\label{ion_fit}
	%\rho(z) = \frac12(\rho_1+\rho_2) - \frac12\left(\rho_1-\rho_2\right)\tanh\left(\frac{z-z_0}{d}\right) + ae^{-\frac{(x-b)^2}{2c^2}}
%\end{equation}

%The Gaussian function allows one to locate the anion peak height $a$, centered at an offset location $b$, with a width $c$.

%\subsection{Water Coordination}
%The coordination of water and distribution of the various coordination types was determined for each interfacial system. Water coordination refers to the hydrogen-bonding structure of water molecules, and is a measure of the number and type of bonds made. A simple naming scheme used to describe each type of water coordination has been developed previously,\cite{Walker2006b} and so that nomenclature will be used in this work. It has been established that certain bonding structures dominate in different regions of the interface, thus each of the water density profiles will be broken down further into component coordination types to show the areas in which the various types are most prevalent. The parameters used to define a hydrogen-bond are taken from a previous work by this group.\cite{Walker2006b} Analysis of water coordination profiles is valuable for comparison to experimental VSF results within the OH-bond stretching region of the vibrational spectrum. These give a molecular-level description of the composition of an interface, and show the various water bonding types that will contribute most to the VSF signal. Additionally, the results of this work are compared to previous studies on the air and salt water interfaces to establish differences due to the addition of the CCl$_4$ organic phase.

%\subsection{Order Parameters}
%One means of describing molecular orientation near to interfaces is by use of orientational order parameters.\cite{Buffeteau2004} This technique has been applied to biaxial molecules such as water at organic interfaces,\cite{Hore2008} and organic molecules to elucidate structuring within the interfacial region.\cite{Hore2007} In this work we compute the two order parameters, S$_1$ and S$_2$, as functions of distance from the Gibbs dividing surface.

%\begin{equation}\label{s1 parameter}
	%S_1 = \frac12\left<3 \cos^2(\theta) - 1\right>
%\end{equation}
%
%\begin{equation}\label{s2 parameter}
	%S_2 = \frac{\left<\sin(\theta)\cos(2\phi)\right>}{\left<\sin(\theta)\right>}
%\end{equation}

%The order parameters are calculated from the Euler angle values of the molecular ``tilt'' $\theta$, and the ``twist'' $\phi$. %The order parameter distributions are further broken down to show the values for individual water coordination types for determination of orientational behavior of each.

\subsection{Molecular Orientation}
% Describe method used to find the bisector and normal orientation histograms
Several methods have been used previously to show molecular orientation profiles of water molecules throughout simulated interfacial regions.\cite{Wick2006c,Thomas2007,Wick2008a,Wick2007,Fan2009,Galamba2008,Ishiyama2007,Hore2007,Hore2008} Studies have utilized various internal coordinate definitions and a number of angle definitions, orientational order parameters, and probability distributions to relate molecular, or averaged, orientations. In this work we have chosen to compute the orientation of water using two vectors that intuitively describe the orientation in space, given the locations of the three atoms comprising the molecule. The molecular bisector, a vector that points along the axis of symmetry of the water molecule from the hydrogen-end to the oxygen, provides directional orientation similar to the water molecule's dipole. A second vector, that is referred to here as the molecular normal vector, is established as the vector pointing normal to the plane formed by the three atoms of the water molecule and establishes its planar ``tilt''. Analyzing the angle made between these two vectors and a given space-fixed reference axis (herein defined as the long-axis of the simulation cells, oriented perpendicular to the interfacial plane and pointing out of the aqueous phase) is a means of finding the orientation of waters within these simulated systems as illustrated in figure \ref{fig:water-angles}. The angle formed between the molecular bisector and the reference axis will hereafter be referred to as $\theta$, and the molecular normal vector as $\phi$. The analysis in this work reports the cosines of these two angles, and because of the symmetry of the water molecule where the hydrogens are not uniquely identified, the cosines of the two angles are limited as follows: $-1\le\cos\theta\le1$ and $0\le\cos\phi\le1$. We report the orientation profiles of $\theta$ and $\phi$ as functions of the distance from the GDS of the interface, as found from the fitting in our density profile analyses.

\begin{figure}[h!]
\begin{center}
	\includegraphics[scale=1.0]{images/water-angles.png}
	\caption{Angles used to define molecular orientation. The system reference (Ref.) axis is that which is perpendicular to the plane of the aqueous-organic interface, and points out from the aqueous phase into the organic one. The molecular bisector vector points from the hydrogen-end of the water to the oxygen end, and orients along the axis of symmetry. The angle it forms with the reference axis is either aligned or anti-aligned such that $-1\le \cos\theta \le 1$. The angle formed between the vector normal to the molecular plane (formed by the three water atoms) and the reference-axis orients the ``twist'' of the molecule such that $0 \le \cos\phi \le 1$, where the water molecular plane is either laying flat on the interface ($\cos\phi=1$), or the water is perpendicular to the interface ($\cos\phi=0$).}
	\label{fig:water-angles}
\end{center}
\end{figure}

%\begin{figure}[h!]
%\begin{center}
%	\includegraphics[scale=1.0]{images/water-angles.png}
%	\caption{Molecular orientation of water molecules. The system axis perpendicular to the aqueous interface is used as a reference for analysis of molecular orientation. The angle formed between the reference axis and the water molecular bisector vector, $\theta$, may vary such that $-1<\cos(\theta)<1$. When $\cos(\theta)$ is 1, the molecule's bisector is perfectly aligned with the reference axis, and $\cos(\theta)=-1$ results from a perfect anti-alignment. A second angle, $\phi$, describes the molecular ``twist'' about the bisector. The symmetry of the water molecule limits the range of $\phi$ to $0<cos(\phi)<1$. $\cos(\phi)=0$ results from a water that lies with its molecular plane perpendicular to the plane of the interface, and $\cos(\phi)=1$ describes a water lying perfectly parallel to the plane of the interface.
%	\label{fig:water-angles}
%\end{center}
%\end{figure}


%insert graphic showing bisector and water-plane normal vectors relative to the system Z-axis

%\subsection{Radial Distribution Functions}

%n studying the water structure near to the interface with an organic phase, the radial distribution functions (RDF) for the water atoms were computed lending another metric of water's structure. The RDFs, $g(r)$, for each system were calculated and normalized to a gas-phase probability of unity at long distances, representing a complete loss of orientational correlation.

\subsection{Computational SFG}
%Describe SFG computational method
A difficult challenge for experimental surface studies is in understanding the vibrational spectroscopy of liquid water. Hydrogen bonding between water molecules causes inter- and intramolecular couplings that lead to broad spectral envelopes, each containing a distribution of water-bonded species. Simulations provide the analytical capacity to relate the broad lineshapes, and the often difficultly-assessed impact of hydrogen bonding as a function of OH vibrational frequency, to microscopic geometries, forces, and environments. In this work we compute the SFG spectra of the interface between the salt solutions and an organic phase to compare with the experimental results of similar systems.\cite{McFearin2009} Where general agreement is found, we extract from the calculations a more microscopic picture of the interface and water's spectroscopic signatures to compliment the picture derived from the experiment.

The computational method used in this work is based on that of Morita and Hynes\cite{Morita2000} as outlined in a previous study by this group utilizing the same technique.\cite{Walker2008} The computational SFG technique has been improved in more recent studies by Morita et al,\cite{Morita2002,Ishiyama2009} and with other enhanced water models. The technique used in this work has been established to recreate qualitatively the experimental spectra to a sufficient degree such that we may draw qualified conclusions about lineshape and intensity. 

%Our present analysis is concerned primarily with the overall intensity and response, and thus we still draw qualitative conclusions based on our computed results, and utilize the simpler computational methods of our previous works.

%The most challenging short-coming of the current technique is that of reproducing accurately the lower-frequency features of the SFG spectra below the 3200\cm region. 
